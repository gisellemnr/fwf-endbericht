\documentclass[a4paper, 11pt]{article}

% Margins
\usepackage[left=2.5cm, right=2.5cm, top=2.5cm, bottom=2cm]{geometry}

% Font sans serif (arial)
\usepackage{helvet}
\renewcommand{\familydefault}{\sfdefault}

% Line spacing
\renewcommand{\baselinestretch}{1.5}

% Section numbering
\renewcommand\thesection{\Roman{section}}
\renewcommand*\thesubsection{\arabic{subsection}}

\usepackage{hyperref}
\usepackage{nopageno}
\usepackage[parfill]{parskip}
\usepackage{amssymb}
\usepackage[usenames,dvipsnames]{xcolor}
\usepackage{array}

\begin{document}

\begin{center}
% 16pt
\fontsize{16pt}{19.2pt}\selectfont
  \textbf{STAND-ALONE PROJECT}\\
  \textbf{FINAL REPORT}
\end{center}
\vspace{1cm}

\makebox[3.5cm][l]{\textbf{Project number}} P-12345-123
\newline

\makebox[3.5cm][l]{\textbf{Project title}\footnotemark}
\footnotetext{Short title in English and German language.}
English title

\makebox[3.5cm][l]{}
Deutsch Titel
\newline

\makebox[3.5cm][l]{\textbf{Project leader}} Leader name

\makebox[3.5cm][l]{\textbf{Project website}\footnotemark}
\footnotetext{Projects that started after January 1, 2009 are encouraged to have a website.}
\url{http://www.projectsite.com}

\newpage

\section{Summary for public relations work}

The project's most significant results (scientific advances) from the project
leader's point of view should be presented on a single page (DIN A4, 11 pt font,
line spacing 1.5) in a way that is comprehensible to the general public. In this
text, it is important to use as few technical terms as possible in order to
ensure that the text is interesting and understandable to people not familiar
with the field. The main point should be mentioned at the very start of the
summary. Please keep descriptions of the issues addressed and results obtained
short and succinct. Possible applications to or implications for social,
cultural, ecological, medical, economic or technological areas should also be
mentioned briefly.

The summary should be submitted both in \textbf{German} and in \textbf{English}.
The summaries will be made available via the FWF's project
database. The  FWF will not edit the summaries, meaning that the authors bear
full responsibility for the content of these texts.

\subsection{Zusammenfassung f\"{u}r die \"{O}ffentlichkeitsarbeit}

\subsection{Summary for public relations work}

\newpage
\section{Brief project report}
\setcounter{subsection}{0}

\begin{itemize}
\item \textbf{To be written in the language of the original application}
\item Target group:  \textbf{peer reviewers}
\item \textbf{Length:}  not to exceed 16,000 characters (without spaces, approx. 6 pages) in total; please
mention each point (all together on 4 pages minimum, 11 pt font, line spacing 1.5, \textbf{no attachments apart from those
mentioned in section III})
\end{itemize}

\subsection{Report on research work}

\subsubsection{Information on the development of the research project}

\begin{itemize}
\item Overall scientific concept and goals;
\item Was there a fundamental change in research orientation between the start and the
end of the project? If so, what form did the change take, and what effect did it
have on the work?
\end{itemize}

\subsubsection{Most important results and brief description of their significance (main
points) with regard to the following:}

\begin{itemize}
\item Contribution to the advancement of the field (e.g. did the results contribute to
increasing the importance of the field? In what way?);
\item Breaking of new scientific / scholarly ground (to what extent and in what
respects?); 
\item Most important hypotheses / research questions developed (what relevance did the
project have for the development of hypotheses / research questions, e.g. were
new hypotheses / research questions developed or old hypotheses disproved?);
\item Development of new methods;
\item Relevance for other (related) areas of science (transdisciplinary issues and
methods).
\end{itemize}

\subsubsection{Information on the execution of the project, use of available funds and
(where appropriate) any changes to the original project plan relating to the
following:}

\begin{itemize}
\item Duration;
\item Use of personnel;
\item Major items of equipment purchased;
\item Other significant deviations\footnote{The decision as to what should be
regarded as a ``significant deviation'' is the responsibility of the project
leader.  As a guideline, any deviation of more than 25\% from the original
financial plan or work schedule should be accounted for.}.
\end{itemize}

\subsection{Personnel development – Importance of the project for the research
careers of those involved (including the project leader)}

\begin{itemize}
\item Brief comments on the project’s effects on the research careers of all project
members, including special qualifications and special possibilities /
opportunities opened up by the project.
\end{itemize}

\subsection{Effects of the project beyond the scientific field}

\begin{itemize}
\item Brief comments on specific effects beyond the research field, including
activities outside the sphere of academia. 
\end{itemize}

\subsection{Other important aspects (examples)}

\begin{itemize}
\item Project-related participation in national and international scientific /
scholarly conferences, list of most important lectures held;
\item Organisation of symposiums and conferences; 
\item Prizes/awards;
\item Any other aspects.
\end{itemize}

\newpage
\section{Attachments}
\setcounter{subsection}{0}

(lists may be as long as required)

\subsection{Scholarly / scientific publications}

Publications may only be listed if they relate directly to the project. \textbf{Up to
three of the most important publications} should be highlighted as such (e.g.
printed in bold letters).

Please note: In accordance with the guidelines of the FWF concerning Open Access, with 
the submission of the final report, \textbf{all peer-reviewed publications that resulted from
the project have to be made openly accessible}
(see: \url{http://www.fwf.ac.at/en/research-funding/open-access-policy/}).
Exceptions to this rule, e.g., if a publication organ explicitly does not permit 
Open Access, must be proven. For projects funded after 1 January 2015, no 
exceptions are possible.

In the interest of the project continuation, it is requested to provide the activation 
within this period. For inquiries relating to the refund of publication costs please contact
Katharina Rieck via: \url{publikationskosten@fwf.ac.at}. Please note that funding for publication
costs can be requested (under the original project number) for up to three years following
completion of a project.

Please indicate at the end of every peer-review publication (in brackets) the
Open Access (OA) type as following: 

\begin{itemize}
\item Gold OA = published in Open Access Journal, with or without an author fee (see
register of all Open Access Journals \url{http://www.doaj.org/})
\item Hybrid OA = published in a subscription journal but Open Access by an author fee
(see \url{http://en.wikipedia.org/wiki/Hybrid\_open\_access\_journal}) 
\item Green OA = self-archived electronic copy of the final ``accepted manuscript''
which might include an embargo period (see:\\
\url{http://www.fwf.ac.at/en/research-funding/open-access-policy/})
\item Other OA = any other type of Open Access
\item No OA = not published Open Access 
\end{itemize}

\subsubsection{Peer-reviewed publications / already published}
(journals, monographs, anthologies, contributions to anthologies, proceedings,
research data, etc.)

Citations should be provided in a \textbf{commonly used format}. For each work, the
publication list \textbf{must mention the following}:

\begin{itemize}
\item Author(s)
\item Title
\item Journal
\item Issue
\item Year
\item Pages
\item DOI or ISBN (for books)
\item If Open Access: URL
\item Open Access (OA) Type
\end{itemize}

\subsubsection{Non peer-reviewed publications / already published}
(journals, monographs, anthologies, contributions to anthologies, research
reports, working papers / preprints, proceedings, research data, etc.)

Citations should be provided in a \textbf{commonly used format}. For each work, the
publication list \textbf{must mention the following}:

\begin{itemize}
\item Author(s)
\item Title
\item Journal
\item Issue
\item Year
\item Pages
\item DOI or ISBN or URL / if applicable
\item Open Access / if applicable
\item Open Access (OA) Type
\end{itemize}

\subsubsection{Planned publications}

(journals, monographs, anthologies, contributions to anthologies, proceedings,
research data, etc.)

\begin{tabular}{|l|l|}
\hline
Author(s) & \\
\hline
Title & \\
\hline
Sources & \\
\hline
URL (if applicable) & \\
\hline
Peer Review & yes $\Box$ \hspace{3cm} no $\Box$ \\
\hline
Status & in press/accepted $\Box$ \hspace{1cm} submitted $\Box$ \hspace{1cm} in preparation $\Box$ \\
\hline
\end{tabular}

\subsection{Most important academic awards}
(Specific academic awards, honours, prizes, medals or other merits)

\begin{tabular}{|p{9cm}|c|}
\multicolumn{1}{c}{\textbf{Name of award}} & \multicolumn{1}{c}{\textbf{n=national / i=international}} \\
\hline
 & \\
\hline
 & \\
\hline
 & \\
\hline
 & \\
\hline
\end{tabular}


\subsection{Information on results relevant to commercial applications}

\begin{itemize}
\item Type of commercial application:
  \begin{enumerate}
    \item Patent
    \item Licensing
    \item Copyrights (e.g. for software; no publications)
    \item Others
  \end{enumerate}
\end{itemize}

\begin{tabular}{|l|l|}
\hline
Type of commercial application & \\
\hline
Subject / title of the invention / discovery & \\
\hline
Short description of the invention / discovery & \\
\hline
Year & \\
\hline
Status & granted $\Box$ \hspace{2cm} pending $\Box$ \\
\hline
Application reference (or patent number) & \\
\hline
\end{tabular}

\subsection{Publications for the general public and other publications}

(Absolute figures, separate reporting of national / international publications)

\begin{itemize}
  \item Type of dissemination activities:
  \begin{enumerate}
    \item Self-authored publications on  the World Wide Web
    \item Editorial contributions in the media (print, radio, TV, www, etc.)
    \item (Participatory) contributions within science communication
    \item Popular science contributions (books, lectures, exhibitions, films, etc.)
  \end{enumerate}
\end{itemize}

\begin{tabular}{|l|c|c|}
\multicolumn{1}{l}{} & \multicolumn{1}{c}{\textbf{National}} & \multicolumn{1}{c}{\textbf{International}} \\
\hline
Self-authored publications on  the www & & \\
\hline
Editorial contributions in the media & & \\
\hline
(Participatory) contributions within science communication & & \\
\hline
Popular science contributions & & \\
\hline
\end{tabular}

\subsection{Development of collaborations}
Indication of the most important collaborations (no more than 5) that took place
(i.e. were initiated or continued) in the course of the project. Please provide
the name of the collaboration partner (name, title, institution) and a few words
about the scientific content. Please \textbf{categorise} each collaboration arrangement
as follows:

\begin{tabular}{|c|c|c|c|p{13cm}|}
\hline
\huge{\textbf{N}} &   &   &   & Nationality of collaboration partner (please use the ISO-3-letter country code) \\
\hline
  & \huge{\textbf{G}} &   &   & \makebox[1.5cm][l]{Gender} \textbf{F} (female) \newline
                                \makebox[1.5cm][l]{}       \textbf{M} (male) \\
\hline
  &   & \huge{\textbf{E}} &   & 
  \makebox[1.4cm][l]{Extent} \textbf{E1} low (e.g. no joint publications, but
  mention in acknowledgements or similar); \newline
  \makebox[1.4cm][l]{} \textbf{E2} medium (collaboration e.g. with occasional joint publications,
  exchange of materials or similar, but no longer-term exchange of personnel);
  \newline
  \makebox[1.4cm][l]{} \textbf{E3} high (extensive collaboration with mutual
  hosting of group members for research stays, regular joint publications, etc.)
  \\
\hline
  &   &   & \huge{\textbf{D}} &  
  \makebox[1.7cm][l]{Discipline} \textbf{W} within the discipline (within the
  same scientific field) \newline
  \makebox[1.7cm][l]{} \textbf{I} interdisciplinary (involving two or more
  disciplines) \newline
  \makebox[1.7cm][l]{} \textbf{T} transdisciplinary (collaborations outside the sciences)
  \\
\hline
\end{tabular}

\begin{tabular}{|c|c|c|c|l|l|}
\multicolumn{1}{c}{\textbf{N}} &
\multicolumn{1}{c}{\textbf{G}} &
\multicolumn{1}{c}{\textbf{E}} &
\multicolumn{1}{c}{\textbf{D}} &
\multicolumn{1}{c}{\textbf{Name}} &
\multicolumn{1}{c}{\textbf{Institution}} \\
\hline
 & & & & & \\
\hline
 & & & & & \\
\hline
 & & & & & \\
\hline
 & & & & & \\
\hline
\end{tabular}
 
\textbf{Note:} General scientific contact and occasional meetings should not be
considered collaborations for the purposes of this report.

\subsection{Development of human resources in the course of the project}

(Absolute figures with an indication of status (in progress / completed)

\textbf{Note:} It is not possible to assign a \emph{venia} thesis / work
(\emph{Habilitation})
to a single project; here it is necessary to mention those \emph{venia} theses for
which the project was important.  A similar caveat applies to Ph.D. and diploma
theses: The FWF does not support thesis work, but instead funds the scientific
work that forms the basis for such theses.

\begin{tabular}{|p{8cm}|c|c|p{0.5cm}|p{0.5cm}|}
\multicolumn{1}{c}{} &
\multicolumn{1}{c}{\textbf{In progress}} &
\multicolumn{1}{c}{\textbf{Completed}} &
\multicolumn{2}{c}{\textbf{Gender}} \\
\multicolumn{3}{c}{} &
\multicolumn{1}{c}{\textbf{f}} &
\multicolumn{1}{c}{\textbf{m}} \\
\hline
Full professorship & & & & \\
\hline
\emph{Venia} thesis (\emph{Habilitation}) / \newline
Equivalent senior scientist qualification & & & & \\
\hline
Postdoc & & & & \\
\hline
Ph.D. theses & & & & \\
\hline
Master's theses & & & & \\
\hline
Diploma theses & & & & \\
\hline
Bachelor's theses & & & & \\
\hline
\end{tabular}

\subsection{Applications for follow-up projects}

(Please indicate the status of each project and the funding organisation)

\subsubsection{Applications for follow-up projects (FWF projects)}

Please indicate the project type (e.g. stand-alone project, SFB, DK, etc.)

\begin{tabular}{|l|l|}
\hline
Project number (if applicable) & \\
\hline
Project type & \\
\hline
Title / subject & \\
\hline
Status & granted $\Box$ \hspace{0.5cm} pending $\Box$ \hspace{0.5cm} in preparation $\Box$ \\
\hline
Application reference (if a patent is applied) & \\
\hline
\end{tabular}

\subsubsection{Applications for follow-up projects (Other national projects)}

(e.g. FFG, CD Laboratory, K-plus centres, funding from the Austrian central bank
[OeNB], Austrian federal government, provincial agencies, provincial government
or similar sources)

\begin{tabular}{|l|l|}
\hline
Funding agency & 
% Please choose an item: 
% FFG
% Federal ministries
% CDG
% Ludwig Bolzmann-Gesellschaft
% OeNB
% \"{O}AW
% Others
\\
\hline
Other national funding agencies & \\
\hline
Project number (if applicable) & \\
\hline
Project type & \\
\hline
Title / subject & \\
\hline
Status & granted $\Box$ \hspace{1.5cm} pending $\Box$ \hspace{1.5cm} in preparation $\Box$ \\
\hline
Total costs (granted) & \\
\hline
\end{tabular}

\subsubsection{Applications for follow-up projects (international projects)} 

(e.g. EU, ERC or other international funding agencies)

\begin{tabular}{|l|l|}
\hline
Country & \\
\hline
Funding agency & 
% Please choose an item: 
% EU
% ERC 
% Other international research councils (such as DFG, SNF, NSF)
% Foundations (e.g. Wellcome Trust, Volkswagenstiftung)
% Others
\\
\hline
Project number (if applicable) & \\
\hline
Project type & \\
\hline
Title / subject & \\
\hline
Status & granted $\Box$ \hspace{1.5cm} pending $\Box$ \hspace{1.5cm} in preparation $\Box$ \\
\hline
Total costs (granted) & \\
\hline
\end{tabular}

\newpage
\section{Cooperation with the FWF}
\setcounter{subsection}{0}

Please rate the following aspects with regard to your interaction with the FWF.
Please provide any \textbf{additional comments (explanations)} on the supplementary sheet
with a reference to the corresponding question/aspect. 

\textbf{Scale:} \newline
\makebox[0.7cm]{\textbf{-2}} highly unsatisfactory  \newline
\makebox[0.7cm]{\textbf{-1}} unsatisfactory  \newline
\makebox[0.7cm]{\textbf{0}}  appropriate  \newline
\makebox[0.7cm]{\textbf{+1}} satisfactory  \newline
\makebox[0.7cm]{\textbf{+2}} highly satisfactory  \newline
\makebox[0.7cm]{\textbf{X}}  not used

\textbf{Rules}

(i.e. guidelines for: funding programme, application, use of resources, reports)

\begin{tabular}{|p{5cm}|p{7cm}|>{\centering\arraybackslash}m{1.2cm}|}
\multicolumn{2}{c}{} & \multicolumn{1}{c}{{\color{gray}\textbf{Rating}}} \\
\hline
\textbf{Application guidelines} & Length & \\
\hline
 & Clarity & \\
\hline
 & Intelligibility & \\
\hline
\end{tabular}

\textbf{Procedures} (submission, review, decision)

\begin{tabular}{|p{5cm}|p{7cm}|>{\centering\arraybackslash}m{1.2cm}|}
\hline
 & Advising & \\
\hline
 & Duration of procedure & \\
\hline
 & Transparency & \\
\hline
\end{tabular}

\textbf{Project support}

\begin{tabular}{|p{5cm}|p{7cm}|>{\centering\arraybackslash}m{1.2cm}|}
\hline
\textbf{Advising} & Availability & \\
\hline
 & Level of detail & \\
\hline
 & Intelligibility & \\
\hline
\end{tabular}

\begin{tabular}{|p{12.4cm}|>{\centering\arraybackslash}m{1.2cm}|}
\hline
\textbf{Financial transactions} \newline
(credit transfers, equipment purchases, personnel management) & \\
\hline
\end{tabular}

\textbf{Reporting / reviewing / exploitation}

\begin{tabular}{|p{5cm}|p{7cm}|>{\centering\arraybackslash}m{1.2cm}|}
\hline
 & Effort & \\
\hline
 & Transparency & \\
\hline
 & Support in PR work / exploitation & \\
\hline
\end{tabular}

\newpage
\textbf{Comments on cooperation with the FWF:}

\end{document}
